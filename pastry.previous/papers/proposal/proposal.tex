\documentclass{article}
\usepackage{amsmath}
\usepackage{amsthm, amssymb}

\usepackage{psfrag,psfig,epsfig,epsf}
\usepackage{graphics}
\usepackage{url}

%\def\thefootnote{\fnsymbol{footnote}}

%\newtheorem{thm}{Theorem}%[section]
%\newtheorem{defn}[thm]{Definition}
%\newtheorem{lemma}[thm]{Lemma}
%\newtheorem{corollary}[thm]{Corollary}
%\newtheorem{conjecture}[thm]{Conjecture}
%\newtheorem{fact}[thm]{Fact}
%\newtheorem{proposition}[thm]{Proposition}
%\newtheorem{prop}[thm]{Proposition}
%\newtheorem{lem}[thm]{Lemma}
%\newtheorem{procedure}[thm]{Procedure}
%\newtheorem{definition}[thm]{Definition}
%\newtheorem{claim}[thm]{Claim}
%\newtheorem{remark}[thm]{Remark}
%\newtheorem{invariant}[thm]{Invariant}
%\newtheorem{recall}[thm]{Recall}
%\newtheorem{problem}[thm]{Problem}

%%%%%%%%%%%%%%%%%%%%%%%%%%%%%%%%%%%%%%%%%%%%%%%%%%%%%%%%%%%%%%%%%


\title{{\large Comp 520}\\Pastry: Project Proposal}

\author{Romer Gil \hspace{8mm} Andrew Ladd \hspace{8mm} Tsuen-Wan Ngan\\
	\normalsize Department of Computer Science, Rice University\\
	\small \tt \{rgil, aladd, twngan\}@cs.rice.edu}
\date{}

\begin{document}
\maketitle
\begin{abstract}
In this project, we design and develop a scalable, distributed 
object location and routing substrate for wide-area peer-to-peer 
applications based on Pastry~\cite{RoD01B}.
Our target is to develop an Internet ready, public domain version of 
Pastry that is free from the intellectual property issues in 
the original implementation.
We also plan to address several drawbacks in the existing system, such as
network failure, a clear API, and an open wire protocol.
\end{abstract}


\section{Problem Statement}
The popularity of file sharing applications like Napster, 
Gnutella, and FreeNet~\cite{Nap,Gnu,CSW00} has brought much 
attention to Internet peer-to-peer systems.
Peer-to-peer systems have many interesting technical aspects 
like decentralized control, self-organization, 
adaptation, and scalability.
These systems can be characterized as distributed systems 
in which all nodes have identical capabilities and 
responsibilities and all communication is symmetric.

A Pastry system is a self-organizing overlay network of nodes, 
where each node routes client requests and interacts with 
local instances of one or more applications.
Any computer that is connected to the Internet and runs the 
Pastry node software can act as a Pastry node, subject 
only to application-specific security policies.  The advantage of the
Pastry network is that each node need only to maintain at most $O(\log
n)$ connections.  This is achieved using fixed length bit string
unique node ids and routing only to nodes within a fixed Hamming
distance.  A routing table is dynamically maintained at each node.

The existing Pastry implementation is closed source.  Development of
applications on top of this platform is hampered by this restriction.
Furthermore, the existing API is not sufficiently well defined. 

There is currently no communication protocol for Pastry in place.  A real
implementation would have to deal with this problem by defining a
protocol for remote communication between nodes.

Pastry is currently justified by experiments done with simulated
networks.  The cost of communication is proportional to the Euclidean
distance between nodes distributed uniformly on the unit square.  The
model for undesired node behaviour is silent failure.  

The model for communication cost is not realistic.  The Internet is
known to havea very different distribution of costs.  It is not obvious how
Pastry can deal with this issue.

Our open implementation must address these issues.  Our API and
protocol must be
clearly and rigorously defined and our code must be well documented
and structured.  Also, we would like to deal with the problem of
hostile (Byzantine) nodes, the problem of locality and the problem of
routing failure at the internet level.


\section{Motivation}

An open implementation of Pastry with clear API and communication
protocol would make an excellent jumping point for the public at large
to begin to develop application layers.  The resolution of the
oustanding implementation issues would contribute to producing a
robust, real world version of this routing scheme.

In general, P2P is an exciting and useful paradigm for distributed
communication and computation.  A system such as Pastry contributes
towards eliminating the implementation barriers for large scale P2P.
For further motivation, see \cite{CSW00, Nap, Gnu}.

\section{Hypothesis}

We believe that the locality claims made in the paper break down for
non-metric communication (or proximity) cost functions.  

The current results, however, are quite plausible.  The network constructions
on the unit square are reminiscent of the randomized analysis of
certain geometric algorithms and random graph results \cite{MR96}. The
intuition is that by minimizing proximity distance over the candidates
for routing, we can locally construct `good' networks with high
probability.  In particular, the expected ratio of routing path cost
to expected optimal cost converges to a constant slightly larger than
$1$.  This kind of argument implicitly relies on the triangle
inequality.  Without this assumption, it may not be that Pastry
produces `good' routing paths.

It is not immediately clear whether for different distributions and
cost functions that an effective scheme for enforcing locality
exists.  There has been some recent work on constructing
pseudo-geometric data structures on non-metric spaces and dynamically
maintaining them given certain assumptions about the notion of
`distance' \cite{GXZ01}.  It may be that such a scheme exists for reasonable models
of node proximity and distribution.

Against hostile nodes \cite{RoD01B} proposes a randomized routing
scheme and against routing failure at the Internet level, an infrequent
multicast is proposed.  We believe that these solutions could be
implemented and that given certain assumptions about the number of
hostile nodes and the frequency of failures could be shown to be
efficient and effective with high probability.

\section{Method}

\begin{figure}[t]
\centerline{\psfig{figure=diagram.eps,height=6.0cm}}
\caption{Our proposed program architecture} 
\label{fig:structure}
\end{figure}

In figure \ref{fig:structure}, we illustrate the various components
of our proposed architecture.  Our development approach will favour
rapid prototyping of the various modules to speed our entry into the test
phases of the various components.  

The final code will be written in {\tt Java 1.3} and tested under
both {\tt Linux} and {\tt Solaris}.  We will make use of {\tt CVS} and
{\tt Javadoc}. This particular order was chosen to parallelize the
implementation as much as possible.


\section{Expected Results}

We expect to complete our implementation and we expect that our
routing performance will be identical to that of Pastry.  We hope that our
network model will be richer and more realistic than that of the existing
work.   In particular, we wish to address some or all the outstanding
issues raised by the previous sections.

\section{Related Work}
A prototype implementation of Pastry was done in~\cite{RoD01B}.
Existing applications built on top of Pastry include 
PAST~\cite{DrR01,RoD01A} and SCRIBE~\cite{RKC01}.
Beside Pastry, there are currently several other peer-to-peer 
systems in use or under development.
The Napster~\cite{Nap} music exchange service provided much 
of the original motivation for peer-to-peer systems, but it 
is not a pure peer-to-peer system because its database is 
centralized. 
The most prominent file sharing facilities include Gnutella~\cite{Gnu} 
and FreeNet~\cite{CSW00}, but they are not guaranteed to 
find an existing object in the network.

The second generation of peer-to-peer routing and 
location schemes include Pastry along with Tapestry~\cite{ZKJ01}, 
Chord~\cite{SMK01}, and CAN~\cite{RFH01}.
They guarantee a definite answer to a query in a bounded number 
of network hops, while retaining the scalability of FreeNet
and the self-organizating properties of both FreeNet and Gnutella.

Pastry and Tapestry bear some similarity by routing based on 
the landmark hierarchy.  They differ in their approach to 
achieving network locality and to supporting replication, 
and Pastry appears to be less complex.
The Chord protocol forwards messages based on numerical 
difference with the destination address and makes no 
explicit effort to achieve good network locality.
CAN, unlike Pastry, has a routing table with a stable size, 
but the number of routing hops grows faster than $\log N$.

\section{Timetable}

Here are our milestones and timetable,

\begin{enumerate}

\item Create initial drafts of the internal APIs between the routing and
protocol components. (Sep. 28th).

\item Draft both ends of the client API and implement them. (Oct. 10th).

\item Design the network protocol and implement it. (Oct. 13th).

\item Develop and test a prototype of the routing
implementation. (Oct. 17th).

\item Write a throwaway protocol implementation which conforms to the
internal API using RMI calls. (Oct. 17th).

\item Test the correctness of the full implementation of the network
protocol. (Oct. 20th).

\item Implement a preliminary test suite for the full prototype.
(Oct. 20th).

\item Run a prototype test. (Oct. 22nd).

\item Modify the test suite to test the full protocol
implementation. (Oct. 31st).

\item Run a full test. (Nov. 2nd).

\item Cycle between feature implementation, test model enhancement
and test runs.  (Nov. 20th).

\end{enumerate}


%\nocite{*}
\bibliographystyle{plain}
\bibliography{pastry}

\end{document}
