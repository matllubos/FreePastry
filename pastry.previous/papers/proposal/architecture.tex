\documentclass{article}
\usepackage{amsmath}
\usepackage{amsthm, amssymb}

\usepackage{psfrag,psfig,epsfig,epsf}
\usepackage{graphics}
\usepackage{url}

\newcommand{\pastry}{{\tt Pastry }}

\title{Comp 520\\Free Pastry: Architecture}

\author{Romer Gil \hspace{8mm} Andrew Ladd \hspace{8mm} Tsuen-Wan Ngan\\
	\normalsize Department of Computer Science, Rice University\\
	\small \tt \{rgil, aladd, twngan\}@cs.rice.edu}
\date{}

\begin{document}

\maketitle

\section*{Introduction}

This document is designed to an architecture proposal for an open
source and free version of the Pastry routing for peer to peer
networking.  This is a revision of our initial proposal which
describes our proposed architecture in detail.  We have taken an
object oriented approach to the project as we are using {\tt Java}.  When
possible, data structures and other code are taken from the standard
libraries.  A current snapshot of code documentation can be found at
\url{www.cs.rice.edu/~aladd/pastry/docs/index.html}.

\section*{Principal Abstractions}

There are several key abstractions our architecture is built from.
Our proposed architecture decouples messaging from the routing
algorithm.

\begin{itemize}
\item A {\em Pastry node} is a single entity in the \pastry network.

\item A {\em node id} is used as a unique identifier for a pastry node.

\item A {\em node id factory} is used to generate node ids.

\item A {\em node handle} is an interface to any \pastry node.  This
provides a consistent interface to the local node, a virtual node, or
a remote node.

\item A {\em message} is a datagram bound for internal delivery.

\item A {\em message receiver} is any entity which is capable of
receiving messages.  The \pastry node and node handles are examples of
this.  Node handles are not local message receivers.

\item An {\em address} is associated with a local message receiver.
Every message has a destination address.

\item The {\em message dispatch} receives messages that arrive a
\pastry node and redirects them to the message receiver bound to an
address. 

\item The {\em message queue} receives messages and forwards them to
the dispatch.

\item A {\em client} is any object which has identified itself to a
\pastry node for the purposes of being able to interface with the
node.  Clients are message receivers and often local.

\item A {\em client interface} is an interface to a \pastry node which
is given to a client by the node.  This interface is only for
messaging.

\item A {\em manager} is a special client which is unique to
a node and is given more interface priviliges than usual.

\item The {\em routing manager} receives all messages which are being
sent to an external \pastry node.  If it decides that the message is meant
for itself, it strips the message of its wrapper and sends it back to
the \pastry node.  Otherwise, the message is sent to the routing
manager of the appropriate node.

\end{itemize}

In general, anything not described in this list is implemented as a
client, message or manager subclass.

\section*{Exensibility of the Routing Algorithm}

Various algorithms for routing can be implemented in our framework.  In
particular, the various managers and data structures do not need to be
fixed in the \pastry framework.  This supports variations on the
algorithm presented in the paper.  The only necessary abstraction is
the routing manager which also receives a distinguished address.  We
will implement various parts of the algorithm described in the paper.
Data structures like the routing table, leaf set and neighbour set
will have their own messages and managers.

\section*{Security}

Security for \pastry is built over our architecture by having certain
clients act as special purpose message dispatches.  For example, onion
routing could be implemented by having a secure routing client.  A
message to route could contain an encrypted submessage and be
addressed to the secure router of the intermediate hop.  The secure
router decrypts it, verifies its authenticity and obtains a similar
message addessed to some further hop down the line.  It signs the
outer message and then sends it back to its \pastry node for routing.
In general, the client, address, and message framework supports a
number of application extensions.


\end{document}